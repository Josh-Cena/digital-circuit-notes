\documentclass[UTF8]{ctexart}
\usepackage{../Zhihu}
\title{数字电路学习笔记(四):逻辑代数系统}
\begin{document}
\maketitle
逻辑代数系统由基本公式、常用公式、基本规则三部分构成。掌握了这些,设计出的电路可以尽可能地简单,减少故障几率和元件使用;编程时,如果掌握了逻辑函数化简,也能增加条件判断式的可读性,避免写出垃圾代码。

\section*{一、基本公式}

正如加减乘除中存在各式运算律一样,逻辑运算也有运算规律。本章中,我们主要考虑与、或、非运算;异或、同或的运算律可以很方便地推导出来。我们从\textbf{常量的运算}开始。

\[\begin{matrix}
    0+0=0,&0+1=1,&1+1=1\\
    0\times 0=0,&0\times 1=0,&1\times 1=1\\
    0'=1,&1'=0
\end{matrix}\]

以上相当于把真值表重新表达了一遍,比较直观。

接下来,我们开始把式子抽象化,把其中一个常量用变量$A$代替,看看\textbf{常量与变量的运算(0律与1律)}:

\[\begin{matrix}
    0+A=A,&1+A=1\\
    0\cdot A=0,&1\cdot A=A
\end{matrix}\]

这些也很好理解,只要把常量运算公式两两合并就可以得到。时刻要提示自己:变量的值只有0与1两种情况。

最后,是变量间的运算,即\textbf{基本运算律}:

\textit{提示:在布尔运算中,运算优先级是 非 > 与 > 或}。

\textbf{(1)交换律、结合律、分配律}

这些公式基本和四则运算中的形式一样:
\[\begin{matrix}
    A+B=B+A,&A\cdot B=B\cdot A\\
    (A+B)+C=A+(B+C),&(AB)\cdot C=A\cdot (BC)\\
    A(B+C)=AB+AC,&A+BC=(A+B)(A+C)
\end{matrix}\]

特别的分配律:由于在布尔运算中,与运算和或运算常常处于可以互换的地位,因此,我们也有$A+BC=(A+B)(A+C)$。它和与对或的分配律非常接近,只是“与”和“或”调换了而已。可以证明一下公式的正确性:

\begin{equation*}
\begin{aligned}
(A+B)(A+C)&=A\cdot A+AB+AC+BC\\ 
&=A+AB+AC+BC\\ 
&=A(1+B+C)+BC\\ 
&=A+BC
\end{aligned}
\end{equation*}

分别用了:与逻辑分配律;重叠律(见(3));与逻辑分配律逆命题;1律。

\textbf{(2)还原律}
\[A''=A\]

很直观,变量两次取反后回到它本身。和语言中的“双重否定”是一个概念。

\textbf{(3)重叠律}
\[A+A=A,\,A\cdot A=A\]

这两个式子比较难理解一点。但毕竟,布尔运算与四则运算中的“加法”“乘法”不是完全一致的——如果写成$A\cup A=A,\,A\cap A=A$,或者\code{A || A == A, A && A == A},表达的意思也是一样的。在学习逻辑运算时,有许多迷惑性的符号——比如“0”与“1”,它们并不存在大小关系,而只是“真”与“假”的对应关系,与数字0和1无关。
\begin{quote}
一道十分有趣的题目:能否通过$A + A = A$推导出$A = 0$?
不能。由于布尔运算中没有“减法”运算,因此不可以把等式两端同时减去$A$。这告诉我们,虽然乍一看形式十分接近,但仍然不能被算术运算中的思维误导。
\end{quote}
\textbf{(4)互补律}
\[A+A'=1,\,A\cdot A'=0\]

因为$A$与$A'$中有且只有一个为1,因此可以回到1与0的运算上来理解。

\textbf{(5)德·摩根定律}

迎接逻辑学中最著名、最经典、最实用的定律:
\[(A+B)'=A'B',\,(AB)'=A'+B'\]

它如此优美,如此简洁地表明了与逻辑和或逻辑相互转化的关系。用非常数学的话说:
\begin{quote}
并集的补集是补集的交集,交集的补集是补集的并集
\end{quote}
或者,用实际生活中的例子——
\begin{quote}
一架飞机能成功降落的前提是前后起落架均已放下,缺一不可。所以,飞机不能降落,是因为“没有既放下前起落架又放下后起落架”,或者说是“没有放下前起落架或者没有放下后起落架”。
\end{quote}
此公式的用途之一是去括号。初学时,常常不自觉地写出$(AB)'=A'B'$这样的式子。但实际上,去掉带取反的括号时,或要变与,与要变或。

还有一个用处是转换与逻辑和或逻辑。如果在电路设计中只能使用“或”和“非”两种逻辑,我们也照样能表示出与逻辑——$AB=(A'+B')'$。就比如之前提到过的,Minecraft游戏中便只有“或”和“非”,但能够创造出所有逻辑元件,原理就在于此。

\section*{二、常用公式}

以上公式都比较简单,使用时局限性也比较大。所以,我们还推导出了一系列常用公式。它们的使用频率要高得多(基础公式中的德·摩根定律除外)。

\textit{注意:在之前的描述中,尽量避免了出现“相加”“乘积”这样的字眼,以防造成误会;但为了叙述方便,接下来会经常出现这些词汇,它们不是一般意义上的“加法”“乘法”,请务必注意。}

\textbf{(1)吸收公式}
\[A+AB=A\]

两项相加,且其中一项($A$)是另一项($AB$)的因式时,另一项中的\textbf{\textit{其他因式}}就被吸收了。

证明:$A+AB=A(1+B)=A$。先提取公因式;再运用1律。

\textit{这里一系列公式的表达都非常简单抽象,而实际运用时,这两项不一定会长成$A$与$AB$的样子,但哪怕是$A(X+Y)M'+A(X+Y)(B+C'+D)M'E'F'G$这种庞大的式子,只要眼光毒辣,也能发现公因式,并化简成$A(X+Y)M'$。}

\textbf{(2)消因子公式}
\[A+A'B=A+B\]

两项相加,且其中一项($A$)\textbf{取反后}是另一项($A'B$)的因式时,另一项中的\textbf{这个相反因式}就被消去了。(不要在意“吸收”“消去”这两个词到底有什么区别,只是为了区分这两个公式而已)

证明:$A+A'B=(A+A')(A+B)=1\cdot (A+B)=A+B$。其中提取公因式一步比较难想到,是证明过程的重点。

\textit{要注意区分吸收公式和消因子公式:一个是相同的因式,一个是相反的因式;一个直接消去两项中较大的一项,一个仅仅去除部分因式。}

\textbf{(3)并项公式}
\[AB+AB'=A\]

两项相加,部分因子相等($A$),剩下的互补($B$与$B'$),那么可以合并成一项公有因子。

证明:$AB+AB'=A(B+B')=A\cdot 1=A$。

\textbf{(4)消项公式}
\[AB+A'C+BC=AB+A'C\]

这个较为复杂:三项相加,两项中部分因子互补($AB$中的$A$与$A'C$中的$A'$),剩下的都是第三项($BC$)的因式(这两项的乘积不一定要和第三项相同,只需都是第三项的部分因式),那么第三项可以消去。

证明:
\begin{equation*}\begin{aligned}
AB+A'C+BC&=AB+A'C+(A+A')BC\\ 
&=AB+A'C+ABC+A'BC\\ 
&=(AB+ABC)+(A'C+A'BC)\\ 
&=AB(1+C)+A'C(1+B)\\ 
&=AB+A'C
\end{aligned}\end{equation*}

\textit{这个公式使用频率不算很高,但它的证明用到了一种很有用的思想——引入冗余项,以进行进一步简化。可以看到,证明开头,逆运用并项公式,创造了一个新的项,随后和另外两项分别合并。除了这种用法,还可以利用重叠律$A=A+A$,重复写入一项,再和其他项化简。这种方法很有用,但需要训练才能掌握。}

\section*{三、基本规则}

何谓基本规则?很难说清楚。但大概地讲,它描述了对等式进行恒等变形的一些方法。

\textbf{(1)代入规则}

将逻辑函数式中任一变量(或函数)用另一变量(或函数)替换,等式仍成立。这类似方程化简中的“换元法”思想。

如果有方程:$F(M(X,Y,Z),B,C,...)=G(M(X,Y,Z),B,C,...)$,其中$F,\,G,\,M$均表示函数,则可以设$A=M(X,Y,Z)$,从而得到$F(A,B,C,\dots)=G(A,B,C,\dots)$。

举个例子,之前提到,常用公式不仅仅局限于两到三个变量的运算,还可以拓展到更多变量,这可以用代入规则解释。以消因子公式为例:如果有$(M+N)'+(M+N)XY$,则可以设$A=(M+N)',\,B=XY$,把式子变形成$A+A'B$,然后运用公式。

\textbf{(2)反演规则}

再看看德·摩根定律:$(AB)'=A'+B',\,(A+B)'=A'B'$,它可以同样通过应用代入规则,拓展到多个变量:$(ABC\dots)'=A'+B'+C'+\dots,\,(A+B+C+\dots)'=A'B'C'\dots$。

注意到,等式的左边是一系列项的积或和,并进行了取反,而等式的右边则是这个反函数的展开。由此,我们可以推导出化简一个函数的反函数——或者叫反演——的规则。这个规则就是反演规则。得到函数的反演式有两步:

\begin{itemize}
\item 加变乘,乘变加——对应德·摩根定律中与和或的转换,并保证运算顺序不变;
\item 对每个量取反(包括变量变为反变量和0变1,1变0),但保留非单变量的非号。
\end{itemize}
比如,$((A+B)C+0)D'\cdot 1$,先变换符号:$((AB)+C\cdot0)+D'+1$,再对变量逐个取反:$((A'B')+C'\cdot1)+D+0$,最后检查运算顺序,并通过添加去除括号调整:$(A'B'+C')\cdot1+D+0$。就这样,得到了原函数的反演函数。

\textbf{(3)对偶规则}

这个规则便是之前提到的“与逻辑和或逻辑常常可以互换”的一个严谨定义。对偶式的得到也有两步:

\begin{itemize}
\item 加变乘,乘变加,保证运算顺序不变;
\item 0变1,1变0。
\end{itemize}

对偶式的性质是:若两个逻辑式$F_1=F_2$成立,则它们的对偶式$F_1^D=F_2^D$也成立。比如由于$A(B+C)=AB+AC$,通过对偶变换就可以得到$A+BC=(A+B)(A+C)$。这一套东西其实比较无聊,也没什么用,但挺神奇的。

对偶规则也是由德·摩根定律推导而来的,具体证明过程比较trivial,不再赘述。其实,对偶式和反演式的本质区别就是没有了“对所有量取反”这一步。但为什么要保留“对常数取反”呢?如果没有这一步——比如,$0+A=A$的对偶等式是$1\cdot A=A$;如果不对常数取反,就有$0\cdot A=A$,显然不成立。

\divider

第三、四章一开始非常难理解,总结一下公式:

\begin{itemize}
\item 0律:$0+A=A,\,0\cdot A=0$
\item 1律:$1+A=1,\,1\cdot A=A$
\item 交换律:$A+B=B+A,\,A\cdot B=B\cdot A$
\item 结合律:$A+(B+C)=(A+B)+C,\,A\cdot(B+C)=(AB)\cdot C$
\item 分配律:$A(B+C)=AB+AC,\,A+BC=(A+B)(A+C)$
\item 还原律:$A''=A$
\item 重叠律:$A+A=A,\,A\cdot A=A$
\item 互补律:$A+A'=1,\,A\cdot A'=0$
\item 德·摩根定律:$(A'B')=A'+B',\,(A+B)'=A'B'$
\item 吸收公式:$A+AB=A$
\item 消因子公式:$A+A'B=A+B$
\item 并项公式:$AB+AB'=A$
\item 消项公式:$AB+A'C+BC=AB+A'C$
\end{itemize}

为了加深理解,可以尝试一下用公式化简这些逻辑函数式。化简要求:写成若干乘积项相加的形式,没有括号,乘积项数量最少,每个乘积项中因子最少。如果思路受阻,可以随时查看上面的公式。

\begin{enumerate}
\item $AB(A+BC)$
\item $A'BC(B+C')$
\item $(AB+A'B'+A'B+AB')'$
\item $(A+B+C')(A+B+C)$
\item $AC+A'BC+B'C+ABC'$
\item $ABD+AB'CD'+AC'DE+AD$
\item $(A'B+AB')C+ABC+A'B'C$
\item $A'(C'D+CD')+BC'D+ACD'+AB'C'D$
\item $(A+A'C)(A+CD+D)$\cite{exercise}
\end{enumerate}

\divider

参考答案:

\begin{enumerate}
\item $AB$
\item $A'BC$
\item $0$
\item $A+B$
\item $AB+C$
\item $AD+AB'C$
\item $C$
\item $CD'+C'D$
\item $A+CD$
\end{enumerate}

\begin{thebibliography}{5}
    \bibitem{exercise} 题目来源:数字电路与逻辑设计,张俊涛著,清华大学出版社2017版
\end{thebibliography}

\end{document}
\documentclass[UTF8]{ctexart}
\usepackage{../Zhihu}
\title{数字电路学习笔记(五):逻辑设计基础}
\begin{document}
\maketitle
马上就要正式进入电路设计了,再来看最后一个知识点:逻辑设计吧。

之前我们花了两章,探讨了逻辑运算是什么,怎么算;但还有最后一个大问题,巧妇难为无米之炊,我们得先有一个逻辑式,才能对它化简,并基于结果做电路设计。所以,如何把实际生活中的问题转化为逻辑函数式呢?

先介绍两种比较标准的函数的形式:

\begin{itemize}
\item 最小项表达式,是若干单项式相加,可以类比成代数式展开后的样子,形式类似$ABC+ABC'+AB'C+A'B'C'$,得名于其因为项中用乘法连接而使得每一项为1的概率都很小。
\item 最大项表达式,是若干多项式相乘,可以类比成代数式因式分解后的样子,形式类似$(A'+B'+C')(A+B+C')(A+B'+C)(A'+B'+C')$,得名于其因为项中加法连接而使得每一项为1的概率都很大。
\end{itemize}

其中实际中用得比较多的是最小项表达式。本文也将在此式的基础上讨论。

进一步地,我们还会再看逻辑函数与其他表现形式的转化关系,并正式介绍两个工具:逻辑图和卡诺图。

\section*{一、从对问题的描述得出函数式}

我们开篇的问题中,就是利用了把问题转化为比较“标准”的逻辑命题,加以处理的。一般地说,通过自然语言表述得出方程,首先要把任务写成“只有...且...时,才...”(最大项表达式)或“只要...或...时,就...”(最小项表达式)的形式。它要求该问题不能包含太复杂的嵌套关系,变量也不能太多,所以局限比较大。但用这种方法得出的函数式往往可以省去化简的步骤。

\section*{二、从真值表得出函数式}

假设我们有这样一个真值表:
\begin{figure}
    \begin{tabular}{|c|c|c|c|c|}\hline\rowcolor{lightgray}
        $A$ & $B$ & $C$ & $D$ & $X$ \\\hline
        0&0&0&0&1\\\hline
        0&0&0&1&0\\\hline
        0&0&1&0&0\\\hline
        0&0&1&1&0\\\hline
        0&1&0&0&0\\\hline
        0&1&0&1&1\\\hline
        0&1&1&0&1\\\hline
        0&1&1&1&0\\\hline
        1&0&0&0&0\\\hline
        1&0&0&1&1\\\hline
        1&0&1&0&1\\\hline
        1&0&1&1&1\\\hline
        1&1&0&0&0\\\hline
        1&1&0&1&0\\\hline
        1&1&1&0&0\\\hline
        1&1&1&1&1\\\hline
    \end{tabular}
\end{figure}

如何方便地写出它的表达式?

先试试把它转述为自然语言后照着写:“只要$A$,$B$,$C$,$D$分别为0000,或者0101,或者0110,或者......时,$X$就为1”。这样,的确可以写出方程。

那么,能不能直接通过真值表写出函数式呢?想一想,之前提到,所有的描述都可以转化为所谓“最小项表达式”,其中每一个项都是一个单项式,$A$、$B$、$C$、$D$分别取原变量或反变量。比如先看第一行,当$ABCD$取0000时,最终结果为1。——这对应了$A'B'C'D'=1$。所以,从真值表写出最小项表达式的方法是先找出所有使得因变量为1的自变量值的组合,再把每一个组合对应的乘积项写出来,每个变量取值为0则写上反变量,1则写上原变量,这样使得取这组值时该项为1,最后把这些项用加法相连即可。

举个例子:我们至今没有推导过异或逻辑如何用与或非表达。现在让我们证明一遍。

\begin{figure}
    \begin{tabular}{|c|c|c|c|c|}\hline\rowcolor{lightgray}
        $A$ & $B$ & $X$ \\\hline
        0&0&0\\\hline
        0&1&1\\\hline
        1&0&1\\\hline
        1&1&0\\\hline
    \end{tabular}
\end{figure}

有了真值表,就可以直接看使得$X=1$的$A$,$B$取值组合,发现有01,10两个;接着,分别写出它们对应的乘积项,即$A'B$和$AB'$;最后把它们连在一起:$AB'+A'B$,即可。

\section*{三、从函数式得出真值表}

要画真值表,首先必须把所有自变量可能的取值都填在前几列中。建议使用二进制排列,即0000,0001,0010……等,不容易缺漏。

接下来,把函数式化成最小项表达式。比如,如果初始式为$X=A(BC'+AC)$,则需要先展开成$X=ABC'+AC$,再进一步写成每一项都含有三个变量的标准形式——具体地说,运用公式$AB=AB(C+C')=ABC+ABC'$,得到$X=ABC'+AB'C+ABC$。最后,在对应的真值表行中“$X$”一栏填上“1”,其他行则填0——此处,三项分别对应110,101,111。

\begin{figure}
    \begin{tabular}{|c|c|c|c|}\hline\rowcolor{lightgray}
        $A$ & $B$ & $C$ & $X$\\\hline
        0&0&0&0\\\hline
        0&0&1&0\\\hline
        0&1&0&0\\\hline
        0&1&1&0\\\hline
        1&0&0&0\\\hline
        1&0&1&1\\\hline
        1&1&0&1\\\hline
        1&1&1&1\\\hline
    \end{tabular}
\end{figure}

如果说这种方法是反向填表,那自然还有另一种正向方法——把每一行的$A$,$B$,$C$的值依次代入函数式,求出$X$的值。这种方法适用于变量较多,而且原式比较简单,如果完全展开很费时间的式子。

\section*{四、逻辑图与函数式的转化}

什么是逻辑图?我们知道,任何一个函数,比如$a=f(x,y,z)$,都可以表示为一个“黑箱”——

\begin{figure}
\begin{tikzpicture}[scale=1]
    \draw (0,0) -- (0,3.6) -- (2.5,3.6) -- (2.5,0) -- cycle;
    \node at (1.25,1.8) {$a=f(x,y,z)$};
    \draw (-0.7,0.9) node[above] {$z$} -- (0,0.9);
    \draw (-0.7,1.8) node[above] {$y$} -- (0,1.8);
    \draw (-0.7,2.7) node[above] {$x$} -- (0,2.7);
    \draw (2.5,1.8) -- (3.2,1.8) node[above] {$a$};
\end{tikzpicture}
\end{figure}

只要输入一个$x$,一个$y$,一个$z$,这个黑箱就会返回一个对应的$a$的值。而如果我们想查看它的内部逻辑,我们可能会看到这样的:

\begin{figure}
\begin{tikzpicture}[scale=1]
    \draw (-0.7,3) node[above] {$x$} -- (0,3);
    \draw (-0.7,2.1) node[above] {$y$} -- (0,2.1);
    \draw (-0.7,1.2) node[above] {$z$} -- (1.4,1.2);
    \draw (0,3.3) -- (0.8,3.3) -- (0.8,1.8) -- (0,1.8) -- cycle;
    \node at (0.4,2.55) {$+$};
    \draw (0.8,2.55) -- (1.4,2.55);
    \draw (1.4,2.9) -- (2.2,2.9) -- (2.2,0.8) -- (1.4,0.8) -- cycle;
    \node at (1.8,1.85) {$\div$};
    \draw (2.2,1.85) -- (2.9,1.85) node[above] {$a$};
\end{tikzpicture}
\end{figure}

“+”把$x$和$y$连接起来,作$m=x+y$运算;“$\div$”又连接了“+”的输出和$z$,作$a=m\div z$运算,并输出$a$。所以,这个图就可以表示$\displaystyle a=\frac{x+y}{z}$。

同样地,还记得七个基本逻辑的逻辑符号么?

\begin{figure}
    \begin{circuitikz}[scale=0.7,transform shape]
        \ctikzset{logic ports=ieee}
        \draw (0,3.6) node[or port] {};
        \draw (2.5,3.6) node[and port] {};
        \draw (5,3.6) node[not port] {};
        \draw (0,1.8) node[xor port] {};
        \draw (2.5,1.8) node[xnor port] {};
        \draw (0,0) node[nand port] {};
        \draw (2.5,0) node[nor port] {};
    \end{circuitikz}
    \caption*{从左至右,从上至下:与、或、非、异或、同或、与非、或非}
\end{figure}

如果我们有式子$X=A'B+AB'$,就可以用同样的思路,连出一个图:

\begin{figure}
    \begin{circuitikz}[scale=0.7,transform shape]
        \draw (4,3) node[and port] (and1) {};
        \draw (4,1) node[and port] (and2) {};
        \draw (6.5,2) node[or port] (or) {};
        \draw (and1.out) |- (or.in 1);
        \draw (and2.out) |- (or.in 2);
        \draw (or.out) -- ++(0.8,0) node[right,scale={1/0.7}] {$X$};
        \draw (and1.in 1)++(-1,0) node[not port] (not1) {};
        \draw (and2.in 2)++(-1,0) node[not port] (not2) {};
        \draw (not1.out) -- (and1.in 1);
        \draw (not2.out) -- (and2.in 2);
        \draw (and1.in 2) -- ++(-2,0) |- (not2.in);
        \draw (and2.in 1) -- ++(-2.5,0) |- (not1.in);
        \draw let \p1 = (and1.in 2), \p2 = (and2.in 2) in ({\x1-2cm},{(\y1+\y2)/2}) 
            to[short,*-] (-1.5,{(\y1+\y2)/2}) node[left,scale={1/0.7}] {$B$};
        \draw let \p1 = (and1.in 1), \p2 = (and2.in 1) in ({\x1-2.5cm},{(\y1+\y2)/2}) 
            to[short,*-] (-1.5,{(\y1+\y2)/2}) node[left,scale={1/0.7}] {$A$};
    \end{circuitikz}
\end{figure}

上半部分,得到$A'B$;下半部分,是$AB'$。两个再用或连接,就有了$A'B+AB'$。这样,便可以把一个函数式直观地表达出来。并且,在实际的电路制作中,这样的设计图也可以成为抽象的逻辑式与实际的电路板间的桥梁。

所以,要想绘出逻辑图,一般来说,只要先理清函数的运算顺序,再把对应的逻辑符号用线连接起来即可。再看一个例子:经典的档案室开门问题,已有函数式:$X=(A+BC)\cdot D'$

一步步看它的运算顺序:$B$与$C$相乘;$A$与乘积相加;$D$的反变量与这个和相乘。最后输出$X$。因此,可以画出对应的逻辑图。

\begin{figure}
    \begin{circuitikz}[scale=0.7,transform shape]
        \ctikzset{logic ports=ieee}
        \node[scale={1/0.7}] at (0,3) (a) {$A$};
        \node[scale={1/0.7}] at (0,2) (b) {$B$};
        \node[scale={1/0.7}] at (0,1) (c) {$C$};
        \node[scale={1/0.7}] at (0,0) (d) {$D$};
        \draw (2.5,1.5) node[and port] (and1) {};
        \draw (5,2.25) node[or port] (or1) {};
        \draw (3,0) node[not port] (not1) {};
        \draw (7.5,1.125) node[and port] (and2) {};
        \draw (and1.out) -| (or1.in 2);
        \draw (a) -| (or1.in 1);
        \draw (b) -| (and1.in 1);
        \draw (c) -| (and1.in 2);
        \draw (d) -- (not1.in);
        \draw (not1.out) -| (and2.in 2);
        \draw (or1.out) -| (and2.in 1);
        \draw (and2.out) -- ++(1,0) node[right,scale={1/0.7}] {$X$};
    \end{circuitikz}
\end{figure}

反向地,如果有了这个逻辑图,就可以通过沿着逻辑图走向分析,最终得出函数式。

还可以发现,逻辑图和计算机中的运算树本质上是相同的。所以,我们可以用中序遍历的思路,写出函数式。

\section*{五、卡诺图}

在最小项表达式的化简中,我们本质上在做什么呢?比如,$AB'C+ABC$——两项中只有一个变量不同,所以变成了$AC$。

如果我们有$n$变量最小项表达式,那么它至多可以有$2^n$项——因为每个变量都会以原变量或反变量出现。而如果原变量表示1,反变量表示0,则每一项都可以对应一个唯一的$n$位二进制码,比如$AB'C'D$就是1010。

而$AB'C,\,ABC$对应的二进制数为101,111,也只差一个数位。那么,能不能用一个表格,可以容纳所有的可能项,并直观地发现这些能够合并的”相邻项“呢?于是,工程师莫里斯·卡诺便创造了卡诺图。

\begin{figure}
\begin{tabular}{rc|c|c|c|}
    \multirow{2}{*}{\backslashbox{CD}{AB}}&\multicolumn{1}{r}{}&\multicolumn{1}{r}{}&\multicolumn{1}{r}{}&\multicolumn{1}{r}{}\\
    &\multicolumn{1}{r}{\makebox[2em]{00}}&\multicolumn{1}{r}{\makebox[2em]{01}}&\multicolumn{1}{r}{\makebox[2em]{11}}
    &\multicolumn{1}{r}{\makebox[2em]{10}}\\\cline{2-5} 
    \multicolumn{1}{r|}{00}&&&&\\\cline{2-5} 
    \multicolumn{1}{r|}{01}&&&&\\\cline{2-5} 
    \multicolumn{1}{r|}{11}&&&&\\\cline{2-5} 
    \multicolumn{1}{r|}{10}&&&&\\\cline{2-5} 
\end{tabular}
\end{figure}

这是卡诺图的一般形式。先把所有相关变量分为数量大致相等的两部分,一部分($AB$)沿行布置,一部分($CD$)沿列布置。再把它们以格雷码编码——00,01,11,10等;这样,每个格子就对应了最小项表达式中的唯一一个项,比如$AB'C'D$就位于第四列,第二行位置,对应AB=10,CD=01。并且注意到,如果两个项可以化简,如$A'BCD+ABCD$:

\begin{figure}
\begin{tabular}{rc|c|c|c|}
    \multirow{2}{*}{\backslashbox{CD}{AB}}&\multicolumn{1}{r}{}&\multicolumn{1}{r}{}&\multicolumn{1}{r}{}&\multicolumn{1}{r}{}\\
    &\multicolumn{1}{r}{\makebox[2em]{00}}&\multicolumn{1}{r}{\makebox[2em]{01}}&\multicolumn{1}{r}{\makebox[2em]{11}}
    &\multicolumn{1}{r}{\makebox[2em]{10}}\\\cline{2-5} 
    \multicolumn{1}{r|}{00}&&&&\\\cline{2-5} 
    \multicolumn{1}{r|}{01}&&&&\\\cline{2-5} 
    \multicolumn{1}{r|}{11}&&\tikzmark{circ1start}&\tikzmark{circ1end}&\\\cline{2-5} 
    \multicolumn{1}{r|}{10}&&&&\\\cline{2-5} 
\end{tabular}

\begin{tikzpicture}[remember picture,overlay]
    \draw[rounded corners=7pt,thick,dashed]
    ([shift={(-\tabcolsep-1em+0.5ex,2ex)}]pic cs:circ1start) rectangle
    ([shift={(\tabcolsep+1em-0.5ex,-0.8ex)}]pic cs:circ1end);
\end{tikzpicture}
\end{figure}

则它们在卡诺图上必定相邻。它背后的原理是当$BCD=111$时,无论$A$为0还是1,结果都为1,所以$A$就成了无关项,可以消去。按照这个道理,只要是圈起的区域是\textbf{大小为$2^n$的矩形},那么都可以化简成一个项:

\begin{figure}
\begin{tabular}{rc|c|c|c|}
    \multirow{2}{*}{\backslashbox{CD}{AB}}&\multicolumn{1}{r}{}&\multicolumn{1}{r}{}&\multicolumn{1}{r}{}&\multicolumn{1}{r}{}\\
    &\multicolumn{1}{r}{\makebox[2em]{00}}&\multicolumn{1}{r}{\makebox[2em]{01}}&\multicolumn{1}{r}{\makebox[2em]{11}}
    &\multicolumn{1}{r}{\makebox[2em]{10}}\\\cline{2-5} 
    \multicolumn{1}{r|}{00}&&&&\\\cline{2-5} 
    \multicolumn{1}{r|}{01}&\tikzmark{circ2start}&&&\\\cline{2-5} 
    \multicolumn{1}{r|}{11}&&&&\tikzmark{circ2end}\\\cline{2-5} 
    \multicolumn{1}{r|}{10}&&&&\\\cline{2-5} 
\end{tabular}

\begin{tikzpicture}[remember picture,overlay]
    \draw[rounded corners=10pt,thick,dashed]
    ([shift={(-\tabcolsep-1em+0.5ex,2ex)}]pic cs:circ2start) rectangle
    ([shift={(\tabcolsep+1em-0.5ex,-0.8ex)}]pic cs:circ2end);
\end{tikzpicture}
\end{figure}

圈起的区域覆盖了所有$A$、$B$的值,所以$A$、$B$都是无关项;同时还覆盖了$CD$=01与11,所以$C$也是无关项,最终结果为$D$。

卡诺图并不是二维的,而是循环的,可以从一个边界来到另一个边界,形成一个空间中闭合的形状。比如:

\begin{figure}
\begin{tabular}{rc|c|c|c|}
    \multirow{2}{*}{\backslashbox{CD}{AB}}&\multicolumn{1}{r}{}&\multicolumn{1}{r}{}&\multicolumn{1}{r}{}&\multicolumn{1}{r}{}\\
    &\multicolumn{1}{r}{\makebox[2em]{00}}&\multicolumn{1}{r}{\makebox[2em]{01}}&\multicolumn{1}{r}{\makebox[2em]{11}}
    &\multicolumn{1}{r}{\makebox[2em]{10}}\\\cline{2-5} 
    \multicolumn{1}{r|}{00}&\tikzmark{startup}&&&\tikzmark{endup}\\\cline{2-5} 
    \multicolumn{1}{r|}{01}&&&&\\\cline{2-5} 
    \multicolumn{1}{r|}{11}&&&&\\\cline{2-5} 
    \multicolumn{1}{r|}{10}&\tikzmark{startdown}&&&\tikzmark{enddown}\\\cline{2-5} 
\end{tabular}

\begin{tikzpicture}[remember picture,overlay]
    \draw[rounded corners=10pt,thick,dashed]
    ([shift={(-\tabcolsep,2ex)}]pic cs:startup) --
    ++(0,-2.8ex) -- 
    ([shift={(\tabcolsep,-0.8ex)}]pic cs:endup) --
    ++(0,2.8ex);
  \draw[rounded corners=10pt,thick,dashed]
    ([shift={(-\tabcolsep,-0.8ex)}]pic cs:startdown) -- 
    ++(0,2.8ex) -- 
    ([shift={(\tabcolsep,2ex)}]pic cs:enddown) --
    ++(0,-2.8ex);
\end{tikzpicture}
\end{figure}

也是可以合并的。

使用卡诺图化简时,先把所有最小项在表中对应的位置打勾,再用圈覆盖这些勾。具体有三个原则:

\begin{itemize}
\item 圈可以相互重叠交错,但必须覆盖所有勾,且不能覆盖空白格,否则会使函数发生改变。
\item 每个圈必须尽量大,这样可以尽可能多地消去无关变量。
\item 圈要尽量少,这样会使最终的项的数量最少,因为每个圈会对应最简式中的一个项。
\end{itemize}

比如,化简函数
\[X=A'B'C'D'+A'B'CD'+A'BC'D+A'BCD'+A'BCD+AB'CD'+ABC'D+ABCD\]
先把所有项填入卡诺图:

\begin{figure}
\begin{tabular}{rc|c|c|c|}
    \multirow{2}{*}{\backslashbox{CD}{AB}}&\multicolumn{1}{r}{}&\multicolumn{1}{r}{}&\multicolumn{1}{r}{}&\multicolumn{1}{r}{}\\
    &\multicolumn{1}{r}{\makebox[2em]{00}}&\multicolumn{1}{r}{\makebox[2em]{01}}&\multicolumn{1}{r}{\makebox[2em]{11}}
    &\multicolumn{1}{r}{\makebox[2em]{10}}\\\cline{2-5} 
    \multicolumn{1}{r|}{00}&$\checkmark$&&&\\\cline{2-5} 
    \multicolumn{1}{r|}{01}&&$\checkmark$&$\checkmark$&\\\cline{2-5} 
    \multicolumn{1}{r|}{11}&&$\checkmark$&$\checkmark$&\\\cline{2-5} 
    \multicolumn{1}{r|}{10}&$\checkmark$&$\checkmark$&&$\checkmark$\\\cline{2-5} 
\end{tabular}
\end{figure}

再把它们全部圈起来,尽量扩大圈的大小和减少圈的数量:

\begin{figure}
\begin{tabular}{rc|c|c|c|}
    \multirow{2}{*}{\backslashbox{CD}{AB}}&\multicolumn{1}{r}{}&\multicolumn{1}{r}{}&\multicolumn{1}{r}{}&\multicolumn{1}{r}{}\\
    &\multicolumn{1}{r}{\makebox[2em]{00}}&\multicolumn{1}{r}{\makebox[2em]{01}}&\multicolumn{1}{r}{\makebox[2em]{11}}
    &\multicolumn{1}{r}{\makebox[2em]{10}}\\\cline{2-5} 
    \multicolumn{1}{r|}{00}&$\checkmark$\tikzmark{up1}&&&\\\cline{2-5} 
    \multicolumn{1}{r|}{01}&&$\checkmark$\tikzmark{circ3start}&$\checkmark$&\\\cline{2-5} 
    \multicolumn{1}{r|}{11}&&$\checkmark$&\tikzmark{circ3end}$\checkmark$&\\\cline{2-5} 
    \multicolumn{1}{r|}{10}&$\checkmark$\tikzmark{left1}\tikzmark{down1}\tikzmark{circ4start}&\tikzmark{circ4end}$\checkmark$&&$\checkmark$\tikzmark{right1}\\\cline{2-5} 
\end{tabular}

\begin{tikzpicture}[remember picture,overlay]
    \draw[rounded corners=10pt,thick,dashed]
    ([shift={(-\tabcolsep-1em+0.5ex,2ex)}]pic cs:circ3start) rectangle
    ([shift={(\tabcolsep+1em-0.5ex,-0.8ex)}]pic cs:circ3end);
    \draw[rounded corners=7pt,thick,dashed,green]
    ([shift={(-\tabcolsep-1em+0.5ex,2ex)}]pic cs:circ4start) rectangle
    ([shift={(\tabcolsep+1em-0.5ex,-0.8ex)}]pic cs:circ4end);
    \draw[rounded corners=7pt,thick,dashed,blue]
    ([shift={(-\tabcolsep-1ex-1em,2ex)}]pic cs:left1) --
    ++(2.5em,0) -- ++(0,-2.8ex) -- ++(-2.5em,0);
    \draw[rounded corners=7pt,thick,dashed,blue]
    ([shift={(-\tabcolsep-1ex+2.1em,2ex)}]pic cs:right1) -- 
    ++(-2.5em,0) -- ++(0,-2.8ex) -- ++(2.5em,0);
    \draw[rounded corners=7pt,thick,dashed,red]
    ([shift={(-\tabcolsep-1ex,2.3ex)}]pic cs:up1) --
    ++(0,-3.2ex) -- ++(1.2em,0) -- ++(0,3.2ex);
    \draw[rounded corners=7pt,thick,dashed,red]
    ([shift={(-\tabcolsep-1ex,-1.1ex)}]pic cs:down1) -- 
    ++(0,3.2ex) -- ++(1.2em,0) -- ++(0,-3.2ex);
\end{tikzpicture}
\end{figure}

圈起来的区域有:(0000,0010), (0010,0110), (1010,0010), (0101,1101,0111,1111)。所以化简之后,剩下的项分别为:$A'B'D'$, $A'CD'$, $B'CD'$, $BD$。所以,我们最终的化简结果便是
\[X={\color{red}{A'B'D'}}+{\color{green}{A'CD'}}+{\color{blue}{B'CD'}}+BD\]

卡诺图的使用需要一定技巧,所以不是非常常用。但是,卡诺图和真值表都可以作为函数的可视化表达。在“时序电路”一部分,我们将会看到它的作用。
\end{document}